\section{Analysis}

\subsection{Figures!!}
\begin{figure}[H]
\begin{minted}
[
    frame=lines,
    framesep=2mm,
    baselinestretch=1.2,
    fontsize=\footnotesize,
    linenos
]
{java}
void testInput(String userInput) //thumbs up for recursion 
    {
        if(userInput == null){
            System.exit(0);
        }

        if(userInput.equals("") | !isNumber(userInput)){
            JOptionPane.showMessageDialog(this.input, "Please enter a number");
            testInput(getInputFromUser());
        }

        if(Integer.parseInt(userInput) < 0 |  Integer.parseInt(userInput) > randomNumbers.length){
            JOptionPane.showMessageDialog(this.input, outOfBounds);
            testInput(getInputFromUser());
        }

        JOptionPane.showMessageDialog(this.input, "Number at index position: " + userInput + " is " + 
        randomNumbers[Integer.parseInt(userInput)]);
        testInput(getInputFromUser());
    }
\end{minted}
    \centering
    \caption{MonoBehaviour.OnMouseDown() built in function to detect mouse clicks. Checking to make sure that the object is not behind a UI element, and opens the menu to delete the object.}
    \label{fig:MouseDown}
\end{figure}


Citation needed \cite{degrauwe1987idac}

